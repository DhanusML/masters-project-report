The Beckman formulation gives a convex function whose minimizers
are the equillibrium path flow (or link flow) solutions
of the traffic equillibrium problem.

For a given link flow $x_{ij}$ for $i,j\in A$, the Beckman function
is defined as

\begin{equation}\label{beck}
	\sum_{i,j\in A}\int_{0}^{x_{ij}}t_{ij}(\omega)d\omega
\end{equation}

Under a mild assumption that the link travel times 
are increasing functions of the corresponding link flows,
the Beckman function is convex. % insert reference to appendix or
								% some other section here

If the path flow vector is $f$ ($\{f_h: h\in H\}$, where $H$ is the
set of all paths), the Beckman function can be written as a function
of $f$ as well because we have
\begin{equation}\label{linktopath}
	x_{ij} = \sum_{(r,s)\in Z^2}\sum_{h\in H_{rs}} f_h \delta_{ij}^h.
\end{equation}

The constraints on the path flow are that they
\emph{must be non-negative} and
\begin{equation}\label{pathconstr}
	\sum_{h\in H_{rs}} f_h = d_{rs} \quad \forall (r,s)\in Z^2.
\end{equation}


The optimization problem therefore becomes:

\begin{equation}\label{optprob}
\begin{split}
	&\min \sum_{i,j\in A}\int_{0}^{x_{ij}}t_{ij}(\omega)d\omega\hfill\\
	\text{where, } \, &x_{ij} =
	\sum_{(r,s)\in Z^2}\sum_{h\in H_{rs}} f_h \delta_{ij}^h.\\
	\text{s.\ t }\, f_h\geq 0\,&\text{ and }\,
	\sum_{h\in H_{rs}}f_h = d_{rs}\,\forall (r,s)\in Z^2.
\end{split}
\end{equation}

This is a convex optimization problem, since the objective is
convex and the constraints are linear. Moreover, the constraints
also satisfy Slater's constraint qualification conditions.
Thus KKT conditions for this problem are both necessary and
sufficient.

\begin{theorem}
	Solution of the optimization problem \eqref{optprob} gives
	the equillibrium path flow vector.
\end{theorem}

\begin{proof}
	From the previous paragraph, $f$ solves \eqref{optprob} if and
	only if it satisfies the KKT conditions. So, we will first write
	down the KKT conditions.

	The Lagrangian is
	\[
		\mathcal{L}(f, \lambda, \mu) =
			\sum_{i,j\in A}\int_{0}^{x_{ij}}t_{ij}(\omega)d\omega
			-\sum_{h\in H} \lambda_h f_h -
			\sum_{(r,s)\in Z^2}\Big(\sum_{h\in H_{rs}} f_h - d_{rs}\Big).
	\]
	The gradient condition gives:
	\[
		\nabla_h \mathcal{L} = \sum_{ij\in A}t_{ij}(x_{ij})\delta_{ij}^h
		- \lambda_h - \mu_{rs} = 0\, \forall \, h\in H_{rs}.
	\]
	From dual feasibility condition, we have $\lambda_h\geq 0$ for
	every path $h\in H_{rs}$. So,
	\[
		\lambda_h = \sum_{ij\in A}t_{ij}(x_{ij})\delta_{ij}^h
		-\mu_{rs} \geq 0.
	\]
	Also by complementary slackness, $\lambda_h f_h = 0$ for
	every path. Thus
	\[
		f_h\Big( \sum_{ij\in A}t_{ij}(x_{ij})\delta_{ij}^h
		-\mu_{rs}\Big) = 0.
	\]
	So, if $f_h\neq 0$, the other term, $\sum_{ij\in A}t_{ij}(x_{ij})\delta_{ij}^h -\mu_{rs}$
	must be zero. Upon careful inspection, the first term in the sum
	is the travel time on path $h$ under the current flow
	conditions.
	Thus
	\[
		\tau_h = \mu_{rs}\, \forall\, h\in H_{rs}.
	\]
	This means every used path between a given origin-destination
	pair must have equal travel time (equal to $\mu_{rs}$).
	The fact that the travel time is minimal is clear from the
	inequality $\lambda_h = \sum_{ij\in A}t_{ij}(x_{ij})\delta_{ij}^h
		-\mu_{rs} \geq 0$.
\end{proof}
