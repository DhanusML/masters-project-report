\section{Representation of Transport Networks}
	The first step in finding equillibrium solution is to
	model the traffic network in a form that can be handled
	mathematically by a computer.
	Traffic network will be represented as a directed graph.
	Nodes in the graph would represent intersections and
	the \emph{edges/arcs/links} connecting the nodes represents
	the roadway connecting them.
	The nodes and arcs in the graph have various attributes
	like travel-time, distance, cost, number of travellers etc..

	An interesting aspect of the transportation network is that
	the travel time along a link depends on the number of
	passengers traversing the link. The more crowded the link
	is, higher is the travel time. The travel time along a
	link could also depend on other factors like length of
	the link and the quality of road, etc.. Thus each link
	in the network has a travel time function associated
	with it, which is a function of the number of passengers in the
	link. Each link would also have a threshold on the maximum
	passengers it can hold. This is called the capacity of the
	link. We would want the travel time function of a link
	to blow up once the link exceeds its capacity. There are
	various functions that are used to model link travel times.
	The one we will be using is the \emph{Bereau of Public Roads}
	function (BPR function), which is given by:
	\[
		t(x) = t_0(1 + b(x/c)^{p})
	\]
	where $t_0$ is the free-flow travel time (travel time when
	the flow is zero). $c$ is the capacity of the link. $b$ and $p$
	are constant parameters that decides the nature of the link.

	Graph will be denotes as $G= G(N,A)$, where $N$ is the
	set of nodes and $A$ the set of links.
	A link from node $i$ to $j$ is represented as a
	tuple $(i,j)$, where $i$ is the head node and
	$j$ is the tail node.
