The greedy method tries to minimize a quadratic approximation of the
Beckman function unlike the other algorithms we have seen so far.
The other key difference between this algorithm and the Gradient
Projection algorithm in the previous section is a greedy inner
loop. Instead of adding new paths to the used paths set in
every iteration, the present algorithm iterates over the current
used paths sets. This brings the flow solution close to equillibrium
with the current set of paths and thus reduces the number of
iterations of the shortest path algorithm required.


\subsubsection*{Quadratic Approximation}
Let us begin with analyzing the quadratic approximation of the
Beckman function.
Consider the Beckman function for the case of a single OD pair,
$(r,s)$.
\[
	z_{rs} = \sum_{ij\in A}\int_{0}^{x_{ij}^- + x_{ij}^{rs}} t_{ij}(\omega)d\omega,
\]
where $x_{ij}^-$ are flows contributed to the link $ij$ due to all
OD pairs other than $r,s$ and $x_{ij}^{rs}$ is the flow contributed
by the OD pair $r,s$.

The second order Taylor expansion of the above function is
\[
	\hat{z}_{rs}(f) \approx z_{rs}(g) +
	\sum_{h\in H_{rs}}\frac{\partial z_{rs}}{\partial f_h}\bigg|_{g}
	(f_h-g_h)+
	\sum_{h\in H_{rs}} \frac{\partial ^2 z_{rs}}{\partial f_h^2}\bigg|_g(f_h-g_h)^2
\]

Introduce constants
\begin{align}
\frac{\partial z_{rs}}{\partial f_h}\big|_{g}&=
\sum_{(i,j)\in A}\delta_{ij}^h t_{ij}(x_{ij}^g)
=: v_h^g \label{eq:v} \\
\frac{\partial ^2 z_{rs}}{\partial f_h^2}\big|_g &=
\sum_{(i,j)\in A} \delta_{ij}^h t'_{ij}(x_{ij}^g)
=: s_h^g\label{eq:s}.
\end{align}

The Taylor expansion can be then simplified to
\[
	\sum_{h\in H_{rs}}\big[(v_h^g - s_h^g g_h)f_h + \frac{1}{2}
	s_h^g f_h^2 + (s_h^g g_h^2 - v_h^g g_h)\big]
\]

Observe that the term $(s_h^g g_h^2 - v_h^g g_h)$ is a constant.
So the quadratic approximation to the original optimization
problem becomes:

\begin{equation}
\begin{split}
        &\min \sum_{h\in H_{rs}}\big[(v_h^g - s_h^g g_h)f_h +
		\frac{1}{2} s_h^g f_h^2\big]\\
		&\text{s.t}\, \sum_{h\in H_{rs}} f_h = d_{rs}
		\text{ and } f_h\geq 0 \, \forall h\in H_{rs}
\end{split}
\end{equation}


\subsubsection*{KKT Conditions}
The Lagrangian of the problem is
\[
\mathcal{L}(f, \lambda, w_{rs}) =
\sum_{h\in H_{rs}}\big[(v_h^g - s_h^g g_h)f_h+
\frac{1}{2} s_h^g f_h^2\big]
-\sum_{h\in H_{rs}}\lambda_h f_h
-w_{rs}\Big(\sum_{h\in H_{rs}}f_h - d_{rs}\Big)
\]

The gradient condition implies
\[
0 = \nabla_h\mathcal{L} = v_h^g + s_h^g(f_h-g_h) - w_{rs}-\lambda_h,
\,\forall\, h\in H_{rs}
\]

From the dual feasibility condition and complementary slackness
conditions, we also have
\begin{align}
v_h^g + s_h^g(f_h-g_h) - w_{rs}\geq 0\label{eq:kkt:1}\\
f_h(v_h^g + s_h^g(f_h-g_h) - w_{rs}) = 0\label{eq:kkt:2}
\end{align}

Similar to what we saw in section \ref{sec:beckman}, $w_{rs}$ is the
shortest path travel time from source $r$ to destination $s$.
Also observe that the other term, $v_h^g + s_h^g(f_h-g_h)$,
is a linear approximation for the travel time at flow $f$,
given the flow solution at flow values $g$.


\subsubsection*{Flow Update}
Suppose we have a feasible flow vector $g$. An iteration
of the algorithm must find a new vector $f$, which is
closer to the optimal point than $g$ is. This section
describes the motivation for chosing the update rule used
in the greedy algorithm.

Before describing the algorithm, let us first introduce few
constants, that would make the presentation of the algorithm
notationally convinient.

\begin{equation}\label{eq:c}
	c_h^g := v^g_h - s_h^g g_h
\end{equation}

If the set of used paths ($f_h>0$ for these paths)
$\hat{H}_{rs}$ is known, the 
conditions \eqref{eq:kkt:1} and \eqref{eq:kkt:2} imply

\begin{equation}\label{eq:gr:temp1}
	v_h^g + s_h^g(f_h-g_h) = \bar{w}_{rs},\quad \forall h\in
	\hat{H}_{rs}.
\end{equation}
The flow conservation constraint now becomes
\begin{equation}\label{eq:gr:temp2}
	\sum_{h\in \hat{H}_{rs}} f_h = d_{rs}.
\end{equation}
Dividing \eqref{eq:gr:temp1} by $s_h^g$ and summing over
$h\in \hat{H}_{rs}$, we get
\begin{equation}\label{eq:w}
	\bar{w}_{rs} = \frac{d_{rs} + \sum_{h\in \hat{H}_{rs}}c_h^g/s_h^g}
	{\sum_{h\in \hat{H}_{rs}}1/s_h^g}
\end{equation}

Observe that $\bar{w}_{rs}$ only depends on the current flow solution
$g$. The value of $\bar{w}_{rs}$ can be now substituted
into \eqref{eq:gr:temp1} to get
\begin{equation}\label{eq:flow}
	f_h = \bar{w}_{rs}/s_h^g - c_h^g/s_h^g.
\end{equation}

So given a set of used paths $H_{rs}$, performing flow update
using \eqref{eq:flow} would bring the flow solution closer
to satisfying the KKT conditions \eqref{eq:kkt:1} and \eqref{eq:kkt:2}.
The pseudocode for this flow update is given in
algorithm \ref{greedy-alg}.


\subsubsection{Implementation}
The pseudocode for the implementation of the greedy algorithm
is given in algorithm \ref{greedy-alg}. This algorithm calls
another function (algorithm \ref{greedy-loop}) which performs
flow updates based on the
update rule suggested in the previous section.

In each iteration of the greedy algorithm, the shortest
path between each OD pair is computed and is pushed
into the used paths set. Flow adjustments are then made
to the used paths sets using the greedy-loop (algorithm \ref{greedy-loop}).
Lines 4 to 12 of algorithm \ref{greedy-alg} does this.
The main reason for the novelty of this algorithm comes
from the lines 14 to 29 of algorithm \ref{greedy-alg}
The quantity $\Delta_{rs}$ (line 19) is computed for each OD pair,
which represents the amount of variation in travel time present
within the set of used paths. If this quantity is high,
path flow adjustments are made within the set so as to equalize the
travel times of the paths in the set of used paths.

The threshold for deciding whether $\Delta_{rs}$ is high enough also
varies across iterations. It is chosen to be equal to $\hbox{RG}/2$.
So for initial iterations, we allow for a larger variation, and
the restriction on the spread of travel time within the used
path set become stricter as the algorithm progresses.


\begin{algorithm}
\caption{GreedyLoop$(H_{rs})$}
\label{greedy-loop}
\begin{algorithmic}[1]

\STATE The current solution is $\{g_h: h\in H_{rs}\}$
\FOR{$h\in H_{rs}$}
	\STATE Compute $v_h^g$ and $s_h^g$ according to \eqref{eq:v}
	\STATE Comput $c_g^g$ according to \eqref{eq:c}
\ENDFOR

\STATE Sort paths $h\in H_{rs}$ accroding to the increasing
order of $c_h^g$.\\i.e., $H_{rs} = \{1,2,3\ldots\}$ with
$c_1^g\leq c_2^g\leq c_3^g\leq \ldots$.

\STATE $B\leftarrow 1/(s_1^g d_{rs})$
\STATE $C\leftarrow c_1^g/(s_1^g d_{rs})$
\STATE $\bar{w}_{rs}\leftarrow (1+C)/B$
\STATE $h\leftarrow 2, \hat{H}_{rs}\leftarrow \{1\}$

\WHILE{$h\leq \abs{H_{rs}}$ and $c_h^g\leq \bar{w}_{rs}$}
	\STATE $C\leftarrow C + c_h^g/(s_h^g d_{rs})$
	\STATE $B\leftarrow B + 1/(s_h^g d_{rs})$
	\STATE $\bar{w}_{rs}\leftarrow (1+C)/B$
	\STATE $\hat{H}_{rs}\leftarrow \hat{H}_{rs}\cup \{h\}$
	\STATE $h\leftarrow h+1$
\ENDWHILE

\FOR{$h\in \hat{H}_{rs}$}
	\STATE $f_h\leftarrow (\bar{w}_{rs} - c_h^g)/s_h^g$
\ENDFOR

\FOR{$h\in H_{rs}\setminus \hat{H}_{rs}$}
	\STATE $f_h\leftarrow 0$
\ENDFOR

\FOR{$h\in H_{rs}$}
	\IF{$f_h\neq g_h$}
		\STATE Update link flow: $x_{ij}\leftarrow x_{ij}+(f_h-g_h)$
		\STATE Update link times and derivatives
	\ENDIF
\ENDFOR

\STATE $H_{rs}\leftarrow \hat{H}_{rs}$
\RETURN $\{f_h: h\in H_{rs}\}$
\end{algorithmic}
\end{algorithm}


\begin{algorithm}
\caption{Greedy$(G)$}
\begin{algorithmic}[1]

\STATE $H_{rs}\leftarrow$ InitPathBasedAlg$(G)$
\STATE Update link flows and derivatives
\STATE Repeat the rest of the code till convergence.
\FOR{$r\in Z$}
	\STATE DIJSKTRA$(G, r)$
	\FOR{$s\in Z$}
		\STATE $\hat{h}\leftarrow$ Shortest path between $r$ and $s$
		\IF{$\hat{h}\notin H_{rs}$}
			\STATE $H_{rs}\leftarrow H_{rs}\cup \{\hat{h}\}$
		\ENDIF
		\STATE $H_{rs}\leftarrow$ GreedyLoop$(H_{rs})$
	\ENDFOR
\ENDFOR

\STATE $I\leftarrow 0$, Max$I\leftarrow 1000$, $FC\leftarrow 0$
\WHILE{$I<$Max$I$}
	\STATE $I\leftarrow I+1$, $FC\leftarrow 0$
	\FOR{each O-D pair $(r,s)$}
		\IF{$I\%100$ is 0}
			\STATE$\Delta_{rs}\leftarrow \max{v_h}-\min{v_h}$, $h\in H_{rs}$
		\ENDIF

		\IF{$\Delta>$RG$^{k-1}/2$}
			\STATE $FC\leftarrow FC+1$
			\STATE $H_{rs}\leftarrow$ GreedyLoop$(H_{rs})$
			\STATE Update link flows, link times and link derivatives
		\ENDIF
	\ENDFOR

	\IF{$FC$ is $0$}
		\STATE \textbf{break}
	\ENDIF
\ENDWHILE

\end{algorithmic}
\end{algorithm}

