In this chapter we will discuss four different algorithms to solve
the Traffic Equillibrium problem. The first two generates
the equillibrium link flows and hence are called link-based
methods. The other two algorithms provide the equillibrium path
flows and are called path based methods.

All four algorithms proceed in a similar way. We start with
an initial feasible flow vector and update this vector until
equillibrium is attained. An important aspect that varies across
the algorithms is how the flow is updated. Different ways of
doing this makes one algorithm more efficient than the other.

Another key point is that due to errors in floating point operations,
it is impossible for the algorithm to converge exactly to the
equillibrium solution. Also for practical applications, it is
enough to obtain a solution which is reasonably close to the
equillibrium. Trying to obtain a higher precision would only
increase the run-time and yield no benefits.
This calls for a metric for how close the current solution is
close to equillibrium. We use \emph{Relativity Gap} to measure this.

\begin{definition}[Relativity Gap]
	Suppose $x$ is the current link flow solution vector. If $\hat{x}$
	is the all-or-nothing flow solution (i.e., all flow is
	assigned to the shortest path between each O-D pair), then
	the \emph{Shortest Path Travel Time}, SPTT is defined as
	\[
		\hbox{SPTT} = \sum_{ij\in A}t_{ij}(x_{ij})\hat{x}_{ij},
	\]
	\emph{Total System Travel Time}, TSTT is defined as
	\[
		\hbox{TSTT} = \sum_{ij\in A} t_{ij}(x_{ij})x_{ij},
	\]
	and finally, the \emph{Relativity Gap}, RG is defined as
	\[
		\hbox{RG} = \frac{\hbox{TSTT}}{\hbox{SPTT}}-1
	\]
\end{definition}

Observe that at equillibrium, RG must be equal to zero since all
used paths have minimal travel time. In the current implementations,
we have used $\hbox{RG}<10^{-4}$ as the stopping condition.


\section{Link Based Methods}
In the link flow methods, the link flows are
initialized to a feasible flow vector. This flow
vector is iteratively updated until equillibrium is attained,
making sure that the updated flows remain feasible
across the iterations.

In the current implementations, the initial feasible flow
vector is chosen to be the all or nothing assignment. The
pseudocode for the initialization is given in Algorithm~\ref{alg:initlink}.

\begin{algorithm}
\caption{InitializeLinkFlowAlg(G)}
\begin{algorithmic}
\FOR{$r\in Z$} %\COMMENT{Initialization}
	\STATE DIJSKTRA$(G,r)$
	\FOR{$s\in Z, (i,j)\in p_{rs}$}
		\STATE $\hat{x}_{ij}\leftarrow d_{rs}$
	\ENDFOR
\ENDFOR
\RETURN $\hat{x}$
\end{algorithmic}
\end{algorithm}



\subsection{Method of Successive Averages (MSA)}
In MSA, the link flows in the next iteration is obtained by taking
the convex combination of the current link flow vector with
the all or nothing link flow solution. The coefficient for
convex combination is the inverse of the iteration number.
This makes sure that the flow vector lies inside the feasible region,
since the feasible region is a convex set.
The pseudocode for MSA is given in algorithm \ref{alg:msa}.

\begin{algorithm}
\caption{MSA$(G)$}
\begin{algorithmic}
\STATE  $\hat{x}\leftarrow$ InitializeLinkFlowAlg(G)

\STATE $k\leftarrow 1$
\STATE Find a feasible $\hat{x}$
\WHILE {RG $> 10^{-4}$}
	\STATE $x\leftarrow \frac{1}{k}\hat{x} + (1-\frac{1}{k})x$
	\STATE Update $t(x)$
	\STATE $\hat{x}\leftarrow 0$
	\FOR{$r\in Z$}
		\STATE DIJSKTRA$(G, r)$
		\FOR{$s\in Z,(i,j)\in p_{rs}^*$}
			\STATE $\hat{x}_{ij} \leftarrow \hat{x}_{ij}+d_{rs}$
		\ENDFOR
	\ENDFOR
	\STATE RG $\leftarrow \frac{\hbox{TSTT}}{\hbox{SPTT}} - 1$
	\STATE $k\leftarrow k+1$
\ENDWHILE
\end{algorithmic}
\end{algorithm}



\subsection{Frank Wolfe}
Frank Wolfe Algorithm is similar to MSA. The link flows are
updated by taking convex combination of the current flow solution
with the all or nothing flow solution. The only difference is
in the choice of the coefficient used in the convex combination.
In Frank Wolfe, the algorithm \ref{alg:bisection} (BISECTION) is
used to obtain a coefficient in each iteration. This function returns
the coefficient for which, the convex combination attains minimum
value of Beckman function. This ensures the best possible descent
in each iteration. The pseudocode for Frank Wolfe algorithm is given
in algorithm \ref{alg:fw}.
\begin{algorithm}
\caption{BISECTION$(G, x,\hat{x})$}
\label{alg:bisection}
\begin{algorithmic}[1]

\STATE $\underline{\eta}\leftarrow 0$
\STATE $\overline{\eta}\leftarrow 1$

\WHILE{$\overline{\eta}-\underline{\eta}>\epsilon$}
	\STATE $\eta \leftarrow \frac{1}{2}(\underline{\eta}+\overline{\eta})$
	\IF{$\sum_{ij\in A} t_{ij}(\eta\hat{x}_{ij}+(1-\eta)x_{ij})>0$}
		\STATE $\overline{\eta}\leftarrow \eta$

	\ELSE
		\STATE $\underline{\eta}\leftarrow \eta$
	\ENDIF
\ENDWHILE
\RETURN $\eta$

\end{algorithmic}
\end{algorithm}


\begin{algorithm}
\caption{FW$(G)$}
\label{alg:fw}
\begin{algorithmic}[1]
\STATE  $\hat{x}\leftarrow$ InitializeLinkFlowAlg(G)

\WHILE {RG $> 10^{-4}$}
	\IF{$k=1$}
		\STATE $\eta\leftarrow 1$
	\ELSE
		\STATE{$\eta \leftarrow$ BISECTION$(G, x, \hat{x})$}
	\ENDIF

	\STATE $x\leftarrow \eta\hat{x} + (1-\eta)x$
	\STATE Update $t(x)$
	\STATE $\hat{x}\leftarrow 0$
	\FOR{$r\in Z$}
		\STATE DIJSKTRA$(G, r)$
		\FOR{$s\in Z,(i,j)\in p_{rs}^*$}
			\STATE $\hat{x}_{ij} \leftarrow \hat{x}_{ij}+d_{rs}$
		\ENDFOR
	\ENDFOR
	\STATE RG $\leftarrow \frac{\hbox{TSTT}}{\hbox{SPTT}} - 1$
\ENDWHILE
\end{algorithmic}
\end{algorithm}



\section{Path Based Methods}
In path flow methods, the path flows are first initialized to a
feasible flow vector. This flow vector iteratively gets updated
till equillibrium is attained, making sure that the flow
vector remains in the feasible region.

Similar to the link based methods, the initial flow vector is
chosen to be the all or nothing solution, where all the demand
gets assigned to the shortest path. The pseudocode for initializing
path based methods is given in algorithm~\ref{alg:initpath}.
\begin{algorithm}
\caption{InitializePathFlowAlg(G)}
\label{alg:initpath}
\begin{algorithmic}[1]
\FOR{$r\in Z$} %\COMMENT{Initialization}
	\STATE DIJSKTRA$(G,r)$
	\FOR{$s\in Z$}
		\STATE $\hat{H}_{rs} = \emptyset$
		\STATE Assign all $d_{rs}$ to $\hat{h}$
		\STATE $\hat{H}\leftarrow \hat{H}\cup \{\hat{h}\}$
	\ENDFOR
\ENDFOR
\RETURN $\hat{H}$
\end{algorithmic}
\end{algorithm}


\subsection{Gradient Projection}
In Gradient Projection method, the path flows are updated by moving
the current path flow solution in the direction of the gradient of the
Beckman function and then projecting the resulting vector back to
the feasible region. The pseudocode for Gradient Projection is
given in algorithm \ref{alg:gp}.
\begin{algorithm}
\caption{GP$(G)$}
\label{alg:gp}
\begin{algorithmic}
\STATE $\hat{H}\leftarrow$ InitPathFlowAlg$(G)$
\WHILE{RG>$10^{-4}$}
	\FOR{$r\in Z$}
		\STATE DIJKSTRA$(G, r)$
		\FOR{$s\in Z$}
			\STATE shortest path $\hat{h}$
			\IF{$\hat{h}\notin \hat{H}_{rs}$}
				\STATE $\hat{H}_{rs}\leftarrow \hat{H}_{rs}\cup\{\hat{h}\}$
			\ENDIF
			\IF{$\hat{H}_{rs}$ is singleton}
				\STATE Set the path flow to $d_{rs}$

			\ELSE
				\FOR{$h\in \hat{H}_{rs}, h\neq \hat{h}$}
					\STATE $\hat{A}\leftarrow \{(i,j): (i,j)\in
					(\hat{h}\cup h)\setminus (\hat{h}\cap h)\}$

					\STATE $f_h\leftarrow f_h -
					\min\bigg\{f_h, \frac{\tau_h - \tau_{\hat{h}}}
					{\sum_{i,j\in \hat{A}} t'_{ij}(x_{ij})}\bigg\}$

					\STATE $f_{\hat{h}}\leftarrow f_{\hat{h}} -
					\min\bigg\{f_{\hat{h}}, \frac{\tau_h - \tau_{\hat{h}}}
					{\sum_{i,j\in \hat{A}} t'_{ij}(x_{ij})}\bigg\}$
				\ENDFOR
			\ENDIF
		\ENDFOR
		\STATE Update link flows, travel times and derivatives
	\ENDFOR
	\STATE Remove paths from $\hat{H}$ with zero flows
	\STATE RG$\leftarrow \frac{\hbox{TSTT}}{\hbox{SPTT}}-1$
\ENDWHILE
\end{algorithmic}
\end{algorithm}



\subsection{Greedy}
The greedy method tries to minimize a quadratic approximation of the
Beckman function unlike the other algorithms we have seen so far.
The other key difference between this algorithm and the Gradient
Projection algorithm in the previous section is a greedy inner
loop. Instead of adding new paths to the used paths set in
every iteration, the present algorithm iterates over the current
used paths sets. This brings the flow solution close to equillibrium
with the current set of paths and thus reduces the number of
iterations of the shortest path algorithm required.


\subsubsection*{Quadratic Approximation}
Let us begin with analyzing the quadratic approximation of the
Beckman function.
Consider the Beckman function for the case of a single OD pair,
$(r,s)$.
\[
	z_{rs} = \sum_{ij\in A}\int_{0}^{x_{ij}^- + x_{ij}^{rs}} t_{ij}(\omega)d\omega,
\]
where $x_{ij}^-$ are flows contributed to the link $ij$ due to all
OD pairs other than $r,s$ and $x_{ij}^{rs}$ is the flow contributed
by the OD pair $r,s$.

The second order Taylor expansion of the above function is
\[
	\hat{z}_{rs}(f) \approx z_{rs}(g) +
	\sum_{h\in H_{rs}}\frac{\partial z_{rs}}{\partial f_h}\bigg|_{g}
	(f_h-g_h)+
	\sum_{h\in H_{rs}} \frac{\partial ^2 z_{rs}}{\partial f_h^2}\bigg|_g(f_h-g_h)^2
\]

Introduce constants
\begin{align}
\frac{\partial z_{rs}}{\partial f_h}\big|_{g}&=
\sum_{(i,j)\in A}\delta_{ij}^h t_{ij}(x_{ij}^g)
=: v_h^g \label{eq:v} \\
\frac{\partial ^2 z_{rs}}{\partial f_h^2}\big|_g &=
\sum_{(i,j)\in A} \delta_{ij}^h t'_{ij}(x_{ij}^g)
=: s_h^g\label{eq:s}.
\end{align}

The Taylor expansion can be then simplified to
\[
	\sum_{h\in H_{rs}}\big[(v_h^g - s_h^g g_h)f_h + \frac{1}{2}
	s_h^g f_h^2 + (s_h^g g_h^2 - v_h^g g_h)\big]
\]

Observe that the term $(s_h^g g_h^2 - v_h^g g_h)$ is a constant.
So the quadratic approximation to the original optimization
problem becomes:

\begin{equation}
\begin{split}
        &\min \sum_{h\in H_{rs}}\big[(v_h^g - s_h^g g_h)f_h +
		\frac{1}{2} s_h^g f_h^2\big]\\
		&\text{s.t}\, \sum_{h\in H_{rs}} f_h = d_{rs}
		\text{ and } f_h\geq 0 \, \forall h\in H_{rs}
\end{split}
\end{equation}


\subsubsection*{KKT Conditions}
The Lagrangian of the problem is
\[
\mathcal{L}(f, \lambda, w_{rs}) =
\sum_{h\in H_{rs}}\big[(v_h^g - s_h^g g_h)f_h+
\frac{1}{2} s_h^g f_h^2\big]
-\sum_{h\in H_{rs}}\lambda_h f_h
-w_{rs}\Big(\sum_{h\in H_{rs}}f_h - d_{rs}\Big)
\]

The gradient condition implies
\[
0 = \nabla_h\mathcal{L} = v_h^g + s_h^g(f_h-g_h) - w_{rs}-\lambda_h,
\,\forall\, h\in H_{rs}
\]

From the dual feasibility condition and complementary slackness
conditions, we also have
\begin{align}
v_h^g + s_h^g(f_h-g_h) - w_{rs}\geq 0\label{eq:kkt:1}\\
f_h(v_h^g + s_h^g(f_h-g_h) - w_{rs}) = 0\label{eq:kkt:2}
\end{align}

Similar to what we saw in section \ref{sec:beckman}, $w_{rs}$ is the
shortest path travel time from source $r$ to destination $s$.
Also observe that the other term, $v_h^g + s_h^g(f_h-g_h)$,
is a linear approximation for the travel time at flow $f$,
given the flow solution at flow values $g$.


\subsubsection*{Flow Update}
Suppose we have a feasible flow vector $g$. An iteration
of the algorithm must find a new vector $f$, which is
closer to the optimal point than $g$ is. This section
describes the motivation for chosing the update rule used
in the greedy algorithm.

Before describing the algorithm, let us first introduce few
constants, that would make the presentation of the algorithm
notationally convinient.

\begin{equation}\label{eq:c}
	c_h^g := v^g_h - s_h^g g_h
\end{equation}

If the set of used paths ($f_h>0$ for these paths)
$\hat{H}_{rs}$ is known, the 
conditions \eqref{eq:kkt:1} and \eqref{eq:kkt:2} imply

\begin{equation}\label{eq:gr:temp1}
	v_h^g + s_h^g(f_h-g_h) = \bar{w}_{rs},\quad \forall h\in
	\hat{H}_{rs}.
\end{equation}
The flow conservation constraint now becomes
\begin{equation}\label{eq:gr:temp2}
	\sum_{h\in \hat{H}_{rs}} f_h = d_{rs}.
\end{equation}
Dividing \eqref{eq:gr:temp1} by $s_h^g$ and summing over
$h\in \hat{H}_{rs}$, we get
\begin{equation}\label{eq:w}
	\bar{w}_{rs} = \frac{d_{rs} + \sum_{h\in \hat{H}_{rs}}c_h^g/s_h^g}
	{\sum_{h\in \hat{H}_{rs}}1/s_h^g}
\end{equation}

Observe that $\bar{w}_{rs}$ only depends on the current flow solution
$g$. The value of $\bar{w}_{rs}$ can be now substituted
into \eqref{eq:gr:temp1} to get
\begin{equation}\label{eq:flow}
	f_h = \bar{w}_{rs}/s_h^g - c_h^g/s_h^g.
\end{equation}

So given a set of used paths $H_{rs}$, performing flow update
using \eqref{eq:flow} would bring the flow solution closer
to satisfying the KKT conditions \eqref{eq:kkt:1} and \eqref{eq:kkt:2}.
The pseudocode for this flow update is given in
algorithm \ref{greedy-alg}.


\subsubsection{Implementation}
The pseudocode for the implementation of the greedy algorithm
is given in algorithm \ref{greedy-alg}. This algorithm calls
another function (algorithm \ref{greedy-loop}) which performs
flow updates based on the
update rule suggested in the previous section.

In each iteration of the greedy algorithm, the shortest
path between each OD pair is computed and is pushed
into the used paths set. Flow adjustments are then made
to the used paths sets using the greedy-loop (algorithm \ref{greedy-loop}).
Lines 4 to 12 of algorithm \ref{greedy-alg} does this.
The main reason for the novelty of this algorithm comes
from the lines 14 to 29 of algorithm \ref{greedy-alg}
The quantity $\Delta_{rs}$ (line 19) is computed for each OD pair,
which represents the amount of variation in travel time present
within the set of used paths. If this quantity is high,
path flow adjustments are made within the set so as to equalize the
travel times of the paths in the set of used paths.

The threshold for deciding whether $\Delta_{rs}$ is high enough also
varies across iterations. It is chosen to be equal to $\hbox{RG}/2$.
So for initial iterations, we allow for a larger variation, and
the restriction on the spread of travel time within the used
path set become stricter as the algorithm progresses.


\begin{algorithm}
\caption{GreedyLoop$(H_{rs})$}
\begin{algorithmic}[1]

\STATE The current solution is $\{g_h: h\in H_{rs}\}$
\FOR{$h\in H_{rs}$}
	\STATE Compute $v_h^g$ and $s_h^g$ according to ---fill----
	\STATE Comput $c_g^g$ according to ---fill----
\ENDFOR

\STATE Sort paths $h\in H_{rs}$ accroding to the increasing
order of $c_h^g$.\\i.e., $H_{rs} = \{1,2,3\ldots\}$ with
$c_1^g\leq c_2^g\leq c_3^g\leq \ldots$.

\STATE $B\leftarrow 1/(s_1^g d_{rs})$
\STATE $C\leftarrow c_1^g/(s_1^g d_{rs})$
\STATE $\bar{w}_{rs}\leftarrow (1+C)/B$
\STATE $h\leftarrow 2, \hat{H}_{rs}\leftarrow \{1\}$

\WHILE{$h\leq \abs{H_{rs}}$ and $c_h^g\leq \bar{w}_{rs}$}
	\STATE $C\leftarrow C + c_h^g/(s_h^g d_{rs})$
	\STATE $B\leftarrow B + 1/(s_h^g d_{rs})$
	\STATE $\bar{w}_{rs}\leftarrow (1+C)/B$
	\STATE $\hat{H}_{rs}\leftarrow \hat{H}_{rs}\cup \{h\}$
	\STATE $h\leftarrow h+1$
\ENDWHILE

\FOR{$h\in \hat{H}_{rs}$}
	\STATE $f_h\leftarrow (\bar{w}_{rs} - c_h^g)/s_h^g$
\ENDFOR

\FOR{$h\in H_{rs}\setminus \hat{H}_{rs}$}
	\STATE $f_h\leftarrow 0$
\ENDFOR

\FOR{$h\in H_{rs}$}
	\IF{$f_h\neq g_h$}
		\STATE Update link flow: $x_{ij}\leftarrow x_{ij}+(f_h-g_h)$
		\STATE Update link times and derivatives
	\ENDIF
\ENDFOR

\STATE $H_{rs}\leftarrow \hat{H}_{rs}$
\RETURN $\{f_h: h\in H_{rs}\}$
\end{algorithmic}
\end{algorithm}


\begin{algorithm}
\caption{Greedy$(G)$}
\begin{algorithmic}[1]

\STATE $H_{rs}\leftarrow$ InitPathBasedAlg$(G)$
\STATE Update link flows and derivatives
\STATE Repeat the rest of the code till convergence.
\FOR{$r\in Z$}
	\STATE DIJSKTRA$(G, r)$
	\FOR{$s\in Z$}
		\STATE $\hat{h}\leftarrow$ Shortest path between $r$ and $s$
		\IF{$\hat{h}\notin H_{rs}$}
			\STATE $H_{rs}\leftarrow H_{rs}\cup \{\hat{h}\}$
		\ENDIF
		\STATE $H_{rs}\leftarrow$ GreedyLoop$(H_{rs})$
	\ENDFOR
\ENDFOR

\STATE $I\leftarrow 0$, Max$I\leftarrow 1000$, $FC\leftarrow 0$
\WHILE{$I<$Max$I$}
	\STATE $I\leftarrow I+1$, $FC\leftarrow 0$
	\FOR{each O-D pair $(r,s)$}
		\IF{$I\%100$ is 0}
			\STATE$\Delta_{rs}\leftarrow \max{v_h}-\min{v_h}$, $h\in H_{rs}$
		\ENDIF

		\IF{$\Delta>$RG$^{k-1}/2$}
			\STATE $FC\leftarrow FC+1$
			\STATE $H_{rs}\leftarrow$ GreedyLoop$(H_{rs})$
			\STATE Update link flows, link times and link derivatives
		\ENDIF
	\ENDFOR

	\IF{$FC$ is $0$}
		\STATE \textbf{break}
	\ENDIF
\ENDWHILE

\end{algorithmic}
\end{algorithm}


