\section{Experiments}
All the algorithms mentioned in the previous chapter were implemented
using python. The stopping criterion for the algorithms were
$\hbox{RG}<10^{-4}$. The algorithms were run on different networks:

\begin{itemize}
	\item Anaheim
	\item Chicago Sketch
	\item Eastern Massachusetts
	\item Sioux Falls
\end{itemize}

Details of the networks are given in table \ref{table:networksdim}
\begin{table}[h!]
\caption{Dimensions of networks}
\label{table:networksdim}
\center
\begin{tabular}{|c|c|c|c|}
\hline
Network	&	Nodes	&	Zones	&	Links\\
\hline
Anaheim		&	416	&	38	&	914\\	
Chicago Sketch	&	933	&	387	&	2950\\
Eastern Massachusetts	&	74	&	74	&	258\\	
Sioux Falls	&	24	&	24	&	76\\
Chicago Regional	&	12982	&	1790	&	39018\\
Bangalore	&	11643	&	534	&	36988\\	
\hline
\end{tabular}
\end{table}



\section{Results}
The run time of the algorithm, is recorded for all four networks.
The Relative Gap after each iteration is also recorded for the algorithms.
Results are given in table \ref{table:resulttable}. The Relativity Gap
against number of iterations plot is given in figure \ref{plots}.

\begin{table}
\caption{Runtime for different algorithms.}
\label{table:resulttable}
\center
\begin{tabular}{|c|c|c|c|c|}
\hline
Network	&	Algorithm	&	Iterations	&	RG	& Time (s)\\
\hline
\multirow{4}{*}{Anaheim}
	&	MSA	&	153	&	$9.89\times 10^{-5}$	&	124.43\\
	&	FW	&	53	&	$9.03\times 10^{-5}$	&	144.95\\
	&	GP	&	2840	&	$9.99\times 10^{-5}$	&	15739.55\\
	&	Greedy	&	5	&	$6.65\times 10^{-5}$	&	39.16\\
	&	Algorithm B	& 6	&	$3.08\times 10^{-5}$	&	0.20\\	
	\hline
\multirow{4}{*}{Chicago Sketch}
	&	MSA		&	670	&	$9.97\times 10^{-5}$	&	$>8$ hours\\
	&	FW		&	93	&	$9.21\times 10^{-5}$	&	$>8$ hours\\
	&	GP		&	80	&	$4.17\times 10^{-3}$	&	$>$8 hours\\
	&	Greedy  &	8	&	$9.45\times 10^{-5}$	&	10,105.92\\
	&	Algorithm B	&	11	&	$9.18\times 10^{-5}$	&	14.86\\
	\hline
\multirow{4}{*}{Eastern Massachusetts}
	&	MSA		&	518	&	$9.77\times 10^{-5}$	&	45.41\\
	&	FW		&	508	&	$9.91\times 10^{-5}$	&	99.38\\
	&	GP		&	13058	&	$9.99\times 10^{-5}$	&	5522.84\\
	&	Greedy	&	7		&	$4.90\times 10^{-5}$	&	3.90\\
	&	Algorithm B & 14	&	$1.44\times 10^{-5}$	&	0.20\\
	\hline
\multirow{4}{*}{Sioux Falls}
	&	MSA		&	7980	&	$9.96\times 10^{-5}$	&	40.72\\
	&	FW		&	1013	&	$9.99\times 10^{-5}$	&	8.47\\
	&	GP		&	1140	&	$9.99\times 10^{-5}$	&	22.58\\
	&	Greedy	&	13		&	$9.20\times 10^{-5}$	&	0.41\\
	&	Algorithm B	&	16	&	$7.38\times 10^{-5}$	&	0.05\\
\hline
\multirow{4}{*}{Chicago Regional}
	& MSA	&	4	&	$1.10$	&	172837\\
	& FW	&	4	&	$4.10\times 10^{-1}$	&	181181\\
	& GP	&	1	&	$3.81\times 10^{-1}$	&	180940\\
	& Greedy	&	1	&	$6.97\times 10^{-2}$	&	183015\\
	& Algorithm B	&	15	&	$6.44\times 10^{-5}$	&	4259\\
\hline
\end{tabular}
\end{table}


\begin{figure}
\caption{$\log$(RG) vs number of iterations plots}
\begin{subfigure}{0.5\linewidth}
\centering
\includegraphics[width=\textwidth]{figures/Anaheim.png}
\caption{Anaheim}
\end{subfigure}
\hfill
\begin{subfigure}{0.5\linewidth}
\includegraphics[width=\textwidth]{figures/ChicagoSketch.png}
\caption{Chicago Sketch}
\end{subfigure}
\hfill
\begin{subfigure}{0.5\linewidth}
\centering
\includegraphics[width=\textwidth]{figures/EMA.png}
\caption{Eastern Massachusetts}
\end{subfigure}
\hfill
\begin{subfigure}{0.5\linewidth}
\centering
\includegraphics[width=\textwidth]{figures/SiouxFalls.png}
\caption{Sioux Falls}
\end{subfigure}
\label{plots}
\end{figure}



\section{Conclusions}
From the experimental data, the greedy algorithm clearly outperforms
other algorithms.
But for large networks the current implementation
failed to give results within reasonable amount of time.
A more efficient implementation using a faster language would
solve this.
